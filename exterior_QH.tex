\documentclass[11pt]{amsart}

\usepackage[usenames,dvipsnames,svgnames,table]{xcolor}
\usepackage[colorlinks=true, pdfstartview=FitV, linkcolor=blue, citecolor=blue, urlcolor=darkblue]{hyperref}

%\usepackage{addfont}
%\addfont{OT1}{rsfs10}{\rsfs}

\usepackage{geometry}                % See geometry.pdf to learn the layout options. There are lots.
\geometry{letterpaper}                   % ... or a4paper or a5paper or ...
%\geometry{landscape}                % Activate for for rotated page geometry
%\usepackage[parfill]{parskip}    % Activate to begin paragraphs with an empty line rather than an indent
\usepackage{graphicx}
\usepackage{mathrsfs}
\usepackage{amssymb}
\usepackage{amsfonts}
\usepackage{mathrsfs}
\usepackage{epstopdf}
\usepackage{lscape}
\usepackage[utf8]{inputenc}
\usepackage{tikz,caption}
\DeclareGraphicsRule{.tif}{png}{.png}{`convert #1 `dirname #1`/`basename #1 .tif`.png}
\usepackage{enumitem}
\setlist[itemize]{leftmargin=2em}
\setlist[enumerate]{leftmargin=2em}
\usepackage{booktabs}
\usepackage{multirow}
\usepackage{mathtools}
\usepackage[linesnumbered,ruled]{algorithm2e}

\definecolor{darkblue}{rgb}{0.0,0,0.7} % darkblue color
\definecolor{darkred}{rgb}{0.7,0,0} % darkred color
\definecolor{darkgreen}{rgb}{0, .6, 0} % darkgreen color

% Dark red emphasis
\newcommand{\defncolor}{\color{darkred}}
\newcommand{\defn}[1]{{\defncolor\emph{#1}}} % emphasis of a definition

\newtheorem{theorem}{Theorem}[section]
\newtheorem{prop}[theorem]{Proposition}
\newtheorem{cor}[theorem]{Corollary}
\newtheorem{lemma}[theorem]{Lemma}
\newtheorem{conj}[theorem]{Conjecture}
\theoremstyle{definition}
\newtheorem{definition}[theorem]{Definition}
\newtheorem{example}[theorem]{Example}
\newtheorem{remark}[theorem]{Remark}
\numberwithin{equation}{section}

\usepackage[colorinlistoftodos]{todonotes}
\newcommand{\idiot}[1]{\vspace{5 mm}\par \noindent
\marginpar{\textsc{Note}}
\framebox{\begin{minipage}[c]{0.95 \textwidth}
#1 \end{minipage}}\vspace{5 mm}\par}
\newcommand{\mike}[1]{\todo[size=\tiny,color=lime!30]{#1 \\ \hfill --- Mike}}
\newcommand{\nantel}[1]{\todo[size=\tiny,color=red!30]{#1 \\ \hfill --- Nantel}}
\newcommand{\yohana}[1]{\todo[size=\tiny,color=Cyan]{#1 \\ \hfill --- Yohana}}
\newcommand{\farad}[2][]{\todo[size=\tiny,color=ForestGreen!30,#1]{#2 \\ \hfill --- Farad}}
\newcommand{\kelvin}[1]{\todo[size=\tiny,color=RoyalBlue!30]{#1 \\ \hfill --- Kel}}

%remove the comment from the following line to remove all the
% extra proofs:
%\renewcommand{\idiot}[1]{}

\title{Quasi-symmetric harmonics of the exterior algebra}
\author{Nantel Bergeron,
Kelvin Chan,
Yohana Solomon,
Farhad Soltani,
Mike Zabrocki}
\date{Draft November 2021}

\begin{document}

\maketitle

\section{Introduction}

\section{Quasisymmetric invariants on the exterior algebra}

Let $R_n = {\mathbb Q}[\theta_1, \theta_2, \ldots, \theta_n]$ be the
polynomial ring in anticommuting variables.
The ring $R_n$ is isomorphic to the exterior algebra of a vector
space of dimension $n$.  The variables of this ring satisfy
\[
\theta_i \theta_j = - \theta_j \theta_i \hbox{ if } 1 \leq i \neq j \leq n
\qquad\hbox{and}\qquad \theta_i^2 = 0 \hbox{ for }1 \leq i \leq n~.
\]
Since a monomial has no repeated variables,
the monomials in $R_n$ are in bijection with subsets of $\{1,2,\ldots, n\}$
and the dimension is therefore equal to $2^n$.

Denote $[n] := \{1,2, \ldots,n\}$ and
let $A = \{a_1 < a_2 < \cdots < a_r \} \subseteq [n]$.
We define $\theta_A := \theta_{a_1} \theta_{a_2} \cdots \theta_{a_r}$,
then the set $\{ \theta_A \}_{A \subseteq [n]}$ is a basis of $R_n$.

We define an action on monomials of $R_n$ and extend this action linearly.
For each integer $1 \leq i < n$, let $\pi_i$ be an operator on $R_n$
that is defined by
\[
\pi_i(\theta_A) = \begin{cases}
\theta_{A} & \hbox{ if } i, i+1 \in A\hbox{ or }i, i+1 \notin A\\
\theta_{A \cup \{i+1\} \backslash \{i\}} & \hbox{ if } i\in A\hbox{ and }i+1 \notin A\\
\theta_{A \cup \{i\} \backslash \{i+1\}} & \hbox{ if } i+1\in A\hbox{ and }i \notin A
\end{cases}~.
\]
These operators instead of exchanging an $i$ for an $i+1$ like the symmetric group
action, have the effect of shifting the indices of the variables (if possible).  They
are therefore known as quasisymmetric operators.  They were studied in depth by
Hivert \cite{H}.  The operators are not multiplicative in general since, for example,
\[
\pi_1( \theta_{1} \theta_{2})
= \theta_1 \theta_2
= - \pi_1( \theta_{1}) \pi_1(\theta_{2})~.
\]

A polynomial that is invariant under the action of quasisymmetric operators
is said to be quasisymmetric invaraint (or just `quasisymmetric').
The quasisymmetric invariants of $R_n$ are
linearly spanned by the elements:
\[
F_{1^r}(\theta_1, \theta_2, \ldots, \theta_n) := \sum_{\substack[1]{A \subseteq [n]\\|A|=r}} \theta_A~.
\]
The notation $F_{1^r}$ for the elements borrow the notation from the
polynomial ring in commuting variable invariants known as the `fundamental
quasisymmetric polynomials'.  The commuting polynomial quasisymmetric
invariants are indexed by compositions.

\subsection{The ideal generated by the quasisymmetric invariants}

Define an ideal of $R_n$ generated by the quasisymmetric invariants as
\[
I_n := \left< F_{1^r}(\theta_1, \theta_2, \ldots, \theta_n) : 1 \leq r \leq n \right>
\]

\begin{remark}
Note that since the operators $\pi_i$ are not multiplicative, it
is unlikely to be the case that $I_n$ as an ideal is invariant
under the action of the $\pi_i$.  Indeed, we find that for $n=4$,
\[
\theta_2 F_{1}(\theta_1, \theta_2, \theta_3, \theta_4) =
-\theta_1 \theta_2 + \theta_2 \theta_3 + \theta_2 \theta_4
\]
and if we apply $\pi_1$ to this element, we obtain
\[
\pi_1(\theta_2 F_{1}(\theta_1, \theta_2, \theta_3, \theta_4)) =
-\theta_1 \theta_2 + \theta_1 \theta_3 + \theta_1 \theta_4
\]
and this can be shown to not be in $I_4$.
\end{remark}

The exterior quasisymmetric coinvariants are defined to be
\mike{I hesitated to give a three letter abbreviated name, like EQC,
but I don't know a better way to
capture the `anti-commuting' and the `quasisymmetric' and the
`coinvariants' in just two letters}
\[
EQC_n := R_n/I_n
\]

\subsection{Differential operators on the exerior algebra}
We can define a set of differential operators on $R_n$ which
will permit us to define the orthogonal complement to the
ideal and a notion of quasisymmetric harmonics.

The operators $\partial_{\theta_i}$ act on monomials in $R_n$
by
\[
\partial_{\theta_i}( \theta_A ) = \begin{cases}
(-1)^{\#\{ j \in A: j<i\}}\theta_{A \backslash \{i\}}&\hbox{ if }i \in A\\
0&\hbox{ if }i \in A
\end{cases}~.
\]

The operators can equally be characterized by their commutation
relations
\[
\partial_{\theta_i} \partial_{\theta_j}\hbox{ if } 1 \leq i \neq j \leq n\qquad
\partial_{\theta_i}^2 = 0\hbox{ for }1 \leq i \leq n
\]

\section{The space of quasisymmetric harmonics of the exterior algebra}

\section{A linear basis of the ideal}

\section{A path model for the quotient}
\end{document}
