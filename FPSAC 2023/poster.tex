\documentclass[20pt, a0paper,landscape]{tikzposter}

\usepackage{amsmath,amssymb,amsthm}
\usepackage[mathcal]{euscript}

\title{Quasisymmetric harmonics of the exterior algebra}
\author{
Nantel Bergeron, %\thanks{\href{mailto:bergeron@yorku.ca}{bergeron@yorku.ca}}, 
Kelvin Chan, %\thanks{\href{mailto:ktychan@yorku.ca}{ktychan@yorku.ca}},
Farhad Soltani, %\thanks{\href{mailto:farhad.soltani91@gmail.com}{farhad.soltani91@gmail.com}}, \and
Mike Zabrocki%\thanks{\href{mailto:zabrocki@yorku.ca}{zabrocki@yorku.ca}}
}
\institute{York University}

\newcommand{\qbinom}[2]{{\left[\begin{smallmatrix} {#1} \\ {#2} \end{smallmatrix} \right]}}

\begin{document}
\maketitle{}

% a bunch of clumns
\begin{columns}
  \column{0.3}
  \block{The Setup}{
    \begin{itemize}
      \item A polynomial ring with anticommuting variables
        \[
          R_{n} = \mathbb{Q}[\theta_{1}, \dots, \theta_{n}].
        \]
      \item Monomials are indexed by sets
        \[
          \theta_{A} = \theta_{i_{1}} \cdots \theta_{i_{k}} \quad\text{where}\quad A = \{ i_{1} < \cdots < i_{k} \}.
        \]
      \item Extend Hivert's $\mathcal{S}_{n}$-action. Act by permutation but ignore signs
        \[
          \sigma(\theta_{A}) = \theta_{\sigma(A)}.
        \]
      \item Invariant polynomials are called \emph{quasisymmetric}. The fundamental ones are
        \[
          \displaystyle F_{1^{k}} = \sum_{A \subseteq \binom{[n]}{k}} \theta_{A}, \quad k = 1,\dots,n.
        \]
      \item The quasisymmetric invariant ideal is $I_{n} = \langle F_{1}, \dots, F_{n} \rangle$. 
    \end{itemize}

    \vspace{1in}

    An example in $R_{3} = \mathbb{Q}[\theta_{1}, \theta_{2}, \theta_{3}]$.
    \begin{itemize}
      \item $\theta_{2}\theta_{1} = - \theta_{1}\theta_{2}$
      \item $(13) \cdot \theta_{2} \theta_{3} = \theta_{(13) \cdot \{2,3\}} = \theta_{\{1,2\}} = \theta_{1} \theta_{2}$ 
      \item $F_{1} = \theta_{1} + \theta_{2} + \theta_{3},\quad F_{11} = \theta_{1}\theta_{2} + \theta_{1}\theta_{3} + \theta_{2}\theta_{3},\quad F_{111} = \theta_{1}\theta_{2}\theta_{3}$.
    \end{itemize}
  }
  
  \column{0.3}
  \block{The Quasisymmetric Polynomials}{
    The algebra $\text{QSym} = \mathbb{Q}[F_{1}, \dots, F_{n}]$ is commutative and
    \[
      F_{1^{r}}F_{1^{s}} = 
      \begin{cases}
        \displaystyle
        \binom{\lfloor\frac{r+s}{2}\rfloor}{\lfloor\frac{r}{2}\rfloor} F_{1^{r+s}} & \text{if } rs \equiv 0 \pmod{2} \\
        0 & \text{otherwise}
      \end{cases}.
    \]

    The ideal $I_{n}$ is actually very simple! 
    \[
      I_{n} = \langle F_{1}, F_{11} \rangle.
    \]
  }

  \column{0.3}
  \block{The Symmetric Polynomials}{
    The ambient ring $R_{n}$ is free over the \emph{symmetric} polynomials. 

    Add example.
  }
\end{columns}

\end{document}

