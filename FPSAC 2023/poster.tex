% NOTE: Final version should be a 4-column layout with 24 pt font.  Since an A0 paper is a 4x4 tiling of A4 paper in the same orientation, it suffices to use a 2-column layout with 12 pt font during editing. Then the column width will look roughly the same.

% uncomment when the contents are ready.
% \documentclass[a0,landscape]{a0poster}

% use this when writing the content.
\documentclass[draft, landscape, 12pt, a4paper]{amsart}

\usepackage[margin=1cm]{geometry}
\usepackage{palatino, parskip, microtype, enumitem}
\usepackage{amsfonts, amsmath, amssymb, amsthm}
\usepackage[mathcal]{euscript}
\usepackage{xcolor, graphicx}

% automatically create multicolumn layout
\usepackage{multicol}
\columnsep = .02\textwidth

\usepackage{ifdraft}

\title{Quasisymmetric harmonics of the exterior algebra}
\author{
  Nantel Bergeron, \thanks{\href{mailto:bergeron@yorku.ca}{bergeron@yorku.ca}}, 
  Kelvin Chan, \thanks{\href{mailto:ktychan@yorku.ca}{ktychan@yorku.ca}},
  Farhad Soltani, \thanks{\href{mailto:farhad.soltani91@gmail.com}{farhad.soltani91@gmail.com}}, \and
  Mike Zabrocki \thanks{\href{mailto:zabrocki@yorku.ca}{zabrocki@yorku.ca}}
}
\date{2023}

% make editing easiser
\usepackage{mdframed}
\newenvironment{postersection}[1][]
{% before 
  \begin{mdframed}[hidealllines=true]
    \IfNoValueTF{#1}{}{
      \begin{mdframed}[backgroundcolor=teal!0,hidealllines=true]
        \centering
        \large #1
      \end{mdframed}
    }
  }
  {% after
  \end{mdframed}
}

% math
\newcommand{\qbinom}[2]{{\left[\begin{smallmatrix} {#1} \\ {#2} \end{smallmatrix} \right]}}

\begin{document}

%--------------------------------------------------
%
% title block
%
%--------------------------------------------------
% takes up the entire top of the page
\begin{mdframed}[backgroundcolor=teal!5, hidealllines=true]
  \centering
  \begin{minipage}[t]{0.6\linewidth}
    {\Large \textcolor{black}{Quaisymmetric harmonics of the exterior algebra}}
    \medskip

    {Nantel Bergeron, Kelvin Chan, Farhad Soltani, Mike Zabrocki} 

    {York University, Toronto, Canada}
  \end{minipage}
  \quad
  \quad
  \begin{minipage}[t]{0.25\linewidth}
    Funding logos
    % NSERC https://www.nserc-crsng.gc.ca/NSERC-CRSNG/acknowledgement_and_logos-mention_et_logos/index_eng.asp
    % YORK  https://www.yorku.ca/brand/py_community_area/brand-building-blocks/logos/    
    % Fields https://www.fields.utoronto.ca/generalinfo/Acknowledging-Fields-Institute-Funding
  \end{minipage}
\end{mdframed}

\medskip
\hrulefill{}

%--------------------------------------------------
%
% content
%
%--------------------------------------------------
\begin{multicols}{\ifdraft{2}{4}}
    \begin{postersection}[The Question]
      What do the coinvariants look like in the setting of the exterior algebra with a quasisymmetric action of $\mathcal{S}_{n}$?

      The short answer: Ballot sequences!
      \[
        Hilb(R_{n}/I_{n}; q) = \sum_{\pi \in NC(n)} q^{\pi}.
      \]
    \end{postersection}

    \begin{postersection}[The History]
      
    \end{postersection}
    
    \begin{postersection}[The Setup]
      Let $R_{n} = \mathbb{Q}[\theta_{1}, \dots, \theta_{n}]$ be a polynomial ring with $\theta_{i} \theta_{j} = - \theta_{j} \theta_{i}$.
      \begin{itemize}
        \item Index monomials by sets $\theta_{\{1,3,5\}} = \theta_{1} \theta_{3} \theta_{5}$.
        \item Extend Hivert's $\mathcal{S}_{n}$-action: Act by permutation but ignore signs
          \[
            (2,5) \cdot \theta_{\{1,3,5\}} = \theta_{\{1,2,3\}} = \theta_{1} \theta_{2} \theta_{3}.
          \]
        \item Invariant polynomials are called \emph{quasisymmetric}. A basis are 
          \begin{align*}
            F_{1^{k}} 
            &=\sum_{\substack{A \subseteq [n]\\|A|=k}} \theta_{A}, \quad k = 1,\dots,n \\
            \intertext{In $\mathbb{Q}[\theta_{1}, \theta_{2}, \theta_{3}]$,}
            F_{1} 
            &= \theta_{1} + \theta_{2} + \theta_{3},\\
            F_{11} 
            &= \theta_{1}\theta_{2} + \theta_{1}\theta_{3} + \theta_{2}\theta_{3},\\
            F_{111} 
            &= \theta_{1}\theta_{2}\theta_{3}
          \end{align*}
        \item The quasisymmetric invariant ideal is $I_{n} = \langle F_{1}, \dots, F_{n} \rangle$. 
      \end{itemize}
    \end{postersection}

    \begin{postersection}[The Structure of $I_{n}$]
      The algebra $\text{QSym} = \mathbb{Q}[F_{1}, \dots, F_{n}]$ is commutative and
      \[
        F_{1^{r}}F_{1^{s}} = 
        \begin{cases}
          \displaystyle
          \binom{\lfloor\frac{r+s}{2}\rfloor}{\lfloor\frac{r}{2}\rfloor} F_{1^{r+s}} & \text{if } rs \equiv 0 \pmod{2} \\
          0 & \text{otherwise}
        \end{cases}.
      \]

      That means the ideal $I_{n}$ is very simple! 
      \[
        I_{n} = \langle F_{1}, F_{11} \rangle.
      \]

      \textbf{Theorem}. The ambient ring $R_{n}$ is free over the \emph{symmetric} polynomials. 
    \end{postersection}


  \end{multicols}
  \end{document}

