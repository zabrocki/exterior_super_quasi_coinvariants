%% if you are submitting an initial manuscript then you should have submission as an option here
%% if you are submitting a revised manuscript then you should have revision as an option here
%% otherwise options taken by the article class will be accepted
\documentclass[submission]{FPSAC2023}
%% but DO NOT pass any options (or change anything else anywhere) which alters page size / layout / font size etc

%% note that the class file already loads {amsmath, amsthm, amssymb}

\newtheorem{thm}{Theorem}
\newtheorem{lem}{Lemma}

\usepackage{lipsum}

%% define your title in the usual way
\title{Quasisymmetric harmonics of the exterior algebra}

%% define your authors in the usual way
%% use \addressmark{1}, \addressmark{2} etc for the institutions, and use \thanks{} for contact details
\author{
Nantel Bergeron\thanks{\href{mailto:bergeron@yorku.ca}{bergeron@yorku.ca}}, 
Kelvin Chan\thanks{\href{mailto:ktychan@yorku.ca}{ktychan@yorku.ca}},
Farhad Soltani\thanks{\href{mailto:farhad.soltani91@gmail.com}{farhad.soltani91@gmail.com}}, \and
Mike Zabrocki\thanks{\href{mailto:zabrocki@yorku.ca}{zabrocki@yorku.ca}}
}

%% then use \addressmark to match authors to institutions here
\address{Dept. of Mathematics and Statistics\\ York  University\\ To\-ron\-to, Ontario M3J 1P3\\ CANADA}
% \thanks{This work is supported in part by York Research Chair and NSERC.  This paper originated in a working session at the Algebraic Combinatorics Seminar at York University}

%% put the date of submission here
\received{\today}

%% leave this blank until submitting a revised version
%\revised{}

%% put your English abstract here, or comment this out if you don't have one yet
%% please don't use custom commands in your abstract / resume, as these will be displayed online
%% likewise for citations -- please don't use \cite, and instead write out your citation as something like (author year)
\abstract{
We study \emph{fermionic} quasisymmetric polynomials in the polynomial ring $R_{n}$ with $n$ anticommuting variables. The main results of this paper are that the quasisymmetric polynomials in $R_n$ form a commutative sub-algebra of $R_n$, there is a basis of the quotient of $R_n$ by the ideal $I_n$ generated by the
quasisymmetric polynomials in $R_n$ that is indexed by ballot sequences, and there is a basis of the ideal generated by quasisymmetric polynomials
that is indexed by sequences that break the ballot condition.
}

%% put your French abstract here, or comment this out if you don't have one
\resume{
Nous étudions les polynômes quasisymétriques fermioniques de l'anneau des polynômes $R_n$ à $n$ variables anti-commutatives. Les principaux résultats de cet article sont : 

\begin{enumerate}
\item Les polynômes quasi-symétriques dans $R_n$ forment une sous-algèbre commutative de $R_n$.

\item Il existe une base du quotient de$R_n$ par l'idéal $I_n$ engendré par les polynômes quasi symétriques dans$R_n$ qui est indexée par des séquences de scrutin. 

\item Il existe une base de l'idéal généré par les polynômes quasi symétriques qui est indexée par des séquences qui brisent la condition de scrutin.
\end{enumerate}
}

%% put your keywords here, or comment this out if you don't have them yet
\keywords{Quasisymmetric Polynomials, Fermionic Variables, Exterior Algebra, Ballot Sequences, Polynomial Harmonics}

%% you can include your bibliography however you want, but using an external .bib file is STRONGLY RECOMMENDED and will make the editor's life much easier
%% regardless of how you do it, please use numerical citations; i.e., [xx, yy] in the text

%% this sample uses biblatex, which (among other things) takes care of URLs in a more flexible way than bibtex
%% but you can use bibtex if you want
\usepackage[backend=bibtex]{biblatex}
\addbibresource{exterior_QH.bib}
%% note the \printbibliography command at the end of the file which goes with these biblatex commands

%%%%%%%%%%%%%%%%%%%%%%%%%%%%%%%%%%%%%%%%%%%%%%%%%%%%%%%%%%%%%%%%%%%%%%%%%%%%%%%%%%%%%%%%%
%%                                                                                     %%
%% NOTE: The rest of the preamble is copied from exterior_QH.tex with minor changes.   %%
%%                                                                                     %%
%%%%%%%%%%%%%%%%%%%%%%%%%%%%%%%%%%%%%%%%%%%%%%%%%%%%%%%%%%%%%%%%%%%%%%%%%%%%%%%%%%%%%%%%%

% NOTE: This is already included in FPSAC class.
% \usepackage[usenames,dvipsnames,svgnames,table]{xcolor}

% NOTE: This is never used.
% \definecolor{darkgreen}{rgb}{0, .6, 0} % darkgreen color

% NOTE: FPSAC class has its own definition of 'urlcolor'. So we don't define ours.
% \definecolor{darkblue}{rgb}{0.0,0,0.7} % darkblue color
% \usepackage[colorlinks=true, pdfstartview=FitV, linkcolor=blue, citecolor=blue, urlcolor=darkblue]{hyperref}

% Dark red emphasis
\definecolor{darkred}{rgb}{0.7,0,0} % darkred color
\newcommand{\defncolor}{\color{darkred}}
\newcommand{\defn}[1]{{\defncolor\emph{#1}}} % emphasis of a definition

% NOTE: I only included the minimum set of packages so that latex compiles.
\usepackage{microtype,tikz,caption,float,amsfonts}
\usepackage{enumitem}
\setlist[itemize]{leftmargin=2em}
\setlist[enumerate]{leftmargin=2em}

\newtheorem{theorem}{Theorem}[section]
\newtheorem{prop}[theorem]{Proposition}
\newtheorem{cor}[theorem]{Corollary}
\newtheorem{lemma}[theorem]{Lemma}
\newtheorem{conj}[theorem]{Conjecture}
\theoremstyle{definition}
\newtheorem{definition}[theorem]{Definition}
\newtheorem{example}[theorem]{Example}
\newtheorem{remark}[theorem]{Remark}
\numberwithin{equation}{section}

\newcommand{\qbinom}[2]{\left[ \begin{smallmatrix}#1\\#2\end{smallmatrix} \right]}

%%%%% you can delete this later
\newcommand{\todo}[1]{\vspace{5 mm}\par \noindent
\marginpar{\textsc{ToDo}}
\framebox{\begin{minipage}[c]{0.95 \textwidth}
\tt #1 \end{minipage}}\vspace{5 mm}\par}


\begin{document}
\maketitle{}

%%%%%%%%%%%%%%%%%%%%%%%%%%%%%%%%%%%%%%
%%%%%%%%%%%%%%%%%%%%%%%%%%%%%%%%%%%%%%
%%%%%%%%%%%%%%%%%%%%%%%%%%%%%%%%%%%%%%
\section{Introduction}
The study of coinvariants of groups dates back to Shephard-Todd and Chevalley~\cite{ST,Chevalley} and has fruitfully produced many connections between algebra, combinatorics and physics.  Motivated by recent developments in coinvariants of symmetric groups and symmetric functions theory incorporating fermionic variables, we study a coinvariant-like quotient of an exterior algebra obtained by the quotient of the ideal generated by quasisymmetric functions in fermionic variables.  The quotient has a dimension that can be interpreted as the number of ballot sequences (or other interpretations, see for instance the OEIS \cite{OEIS} sequences \href{https://oeis.org/A008315}{A008315} and \href{https://oeis.org/A001405}{A001405}).

A notable feature of many quotients similar to coinvariants is their amenability to combinatorial methods.  One well-known example is the coinvariant ring of the symmetric group.  It is the quotient of the polynomial ring $\mathbb{Q}[x_{1},\dots,x_{n}]$ in commuting variables by the ideal generated by the symmetric polynomials with no constant term.  As an $\mathcal{S}_{n}$ representation, this quotient is naturally graded and is well-known to be isomorphic to the regular representation.  Many useful bases of this space have been found by studying combinatorics related to permutations.  For more details, see the nice surveys of \cite{B, GH, MacSchub, Manivel}.

This line of inquiry inspired Garsia and Haiman \cite{GH96,H} to consider the ring of diagonal harmonics, a similar quotient in two sets of commuting variables as an $\mathcal{S}_{n}$ module.  Haiman's work \cite{H2} showed that the diagonal harmonics have a deep connection to the theory of Macdonald polynomials.  A combinatorial expression for the Frobenius image of the diagonal harmonics known as the Shuffle Conjecture \cite{HHLRU} showed that the module structure is closely related to the combinatorics of parking functions and can be described in terms of certain labelled Catalan paths.  This connection relating the symmetric functions and the combinatorial expression was proven in \cite{CM} and is now known as the Shuffle Theorem.

The connection between the combinatorics and the symmetric function expressions of the Shuffle Theorem have been generalized \cite{HRW} and proven \cite{DM} to an expression known as the Delta Conjecture.  The last author with the group at the Fields Institute \cite{Z} proposed a deformation of diagonal harmonics to two sets of commuting variables and one set of anticommuting variables. In this case, the connection of representation theoretic interpretation to the symmetric function expression remains open.  The symmetric function expressions and representation theoretic interpretation was extended further to include the quotient of two sets of commuting and two sets of anticommuting variables in \cite{DIW} to what is known as the Theta Conjecture.  At present, this also remains an open conjecture, but progress has been made on some special cases \cite{IRR, KR, SW, SW2}.

The ring of quasisymmetric polynomials $QSym_{n}$ contains the ring of symmetric polynomials $Sym_{n}$.  Many combinatorial structures of $QSym_{n}$ parallel that of $Sym_{n}$.  Hivert described a Temperley-Lieb $TL_{n}$ action on $\mathbb{Q}[x_{1},\dots,x_{n}]$ making $QSym_{n}$ exactly its trivial representation \cite{Hi}.  In 2003, Aval, F. Bergeron, and the first author studied $QSym$ coinvariant spaces obtained by replacing the ideal of non-constant symmetric functions with the ideal of non-constant quasisymmetric functions \cite{AB,ABB}.  Surprisingly they found that dimensions of $QSym$ coinvariants are equal to the Catalan numbers.  At the heart of their argument is a recursion built from Catalan paths.  Li extended this argument to study some components of coinvariant spaces of diagonally quasisymmetric functions \cite{L}.

Motivated by physics, Desrosiers, Lapointe, and Mathieu \cite{DLM,DLM2} introduced symmetric functions with one set of commuting and one set of anticommuting variables known as symmetric function in superspace.
The commuting variables encode bosons while the anticommuting ones encode fermions, hence the anticommuting variables are sometimes referred to as ``fermionic variables.''  The Hopf algebra structure of the ring of symmetric functions in superspace was extended to quasisymmetric functions in superspace \cite{FLP} and so a natural question is to extend the study of coinvariants of polynomial rings with commuting and anticommuting variables to the quotients of  these polynomial rings by the ideal generated by ``super'' quasisymmetric polynomials.

Parallel to the Delta Conjecture or Theta Conjecture, one ideally would like to understand quasisymmetric coinvariants in multiple sets of commuting and anticommuting variables.  Our study of quasisymmetric coinvariant spaces in one set of anticommuting variables is a first step in that study.  We denote polynomials in anticommuting variables by $R_n$. The main results of this paper show the following interesting facts about symmetric and quasisymmetric functions in anticommuting variables:
\begin{enumerate}
\item The quasisymmetric polynomials in $R_n$ form a commutative sub-algebra of $R_n$ (Proposition~\ref{prop:comm}).
\item That $R_n$ is free over the ring of symmetric polynomials (Proposition \ref{prop:free}).
\item There is a basis of the quotient of $R_n$ by the ideal $I_n$ generated by the
quasisymmetric polynomials in $R_n$ that is indexed by ballot sequences (Proposition \ref{prop:harmbasis}).
The Hilbert series of the quotient is given by
\begin{equation}
    \label{eq:hilb}
   \text{Hilb}_{R_n/I_n}(q) = \sum_{k=0}^{\lfloor{n/2}\rfloor} f^{(n-k,k)} q^k\,,
\end{equation}
 where $f^{(n-k,k)}$ is the  number  of standard tableaux of shape $(n-k,k)$ (Corollary~\ref{cor:hilb}).
\item There is a basis of the ideal generated by quasisymmetric polynomials
that is indexed by sequences that break the ballot condition
(Theorem \ref{thm:basisofideal}) and a minimal Gr\"obner basis
that is a subset of this basis (Corollary \ref{cor:minimalGB}).
\end{enumerate}

\subsection{Acknowledgement} We are grateful to Yohana Solomon for her participation and discussions during this project.



%%%%%%%%%%%%%%%%%%%%%%%%%%%%%%%%%%%%%%
%%%%%%%%%%%%%%%%%%%%%%%%%%%%%%%%%%%%%%
%%%%%%%%%%%%%%%%%%%%%%%%%%%%%%%%%%%%%%
\section{Quasisymmetric invariants on the exterior algebra}

Fix $n$ a positive integer and
let $R_n = {\mathbb Q}[\theta_1, \theta_2, \ldots, \theta_n]$ be the
polynomial ring in anticommuting variables.
The ring $R_n$ is isomorphic to the exterior algebra of a vector
space of dimension $n$.  The variables of this ring satisfy the relations
\[
\theta_i \theta_j = - \theta_j \theta_i \hbox{ if } 1 \leq i \neq j \leq n
\qquad\hbox{and}\qquad \theta_i^2 = 0 \hbox{ for }1 \leq i \leq n~.
\]
Since these conditions impose that a monomial in $R_n$ has no repeated variables,
the monomials are in bijection with subsets of $\{1,2,\ldots, n\}$
and the dimension of $R_n$ is therefore equal to $2^n$.

Denote $[n] := \{1,2, \ldots,n\}$ and
let $A = \{a_1 < a_2 < \cdots < a_r \} \subseteq [n]$.
We define $\theta_A := \theta_{a_1} \theta_{a_2} \cdots \theta_{a_r}$,
then the set of monomials $\{ \theta_A \}_{A \subseteq [n]}$ is a basis of $R_n$.

We define an action on monomials of $R_n$ and extend this action linearly.
For each integer $1 \leq i < n$, let $\pi_i$ be an operator on $R_n$
that is defined by
\begin{equation}\label{eq:pi}
\pi_i(\theta_A) = \begin{cases}
\theta_{A} & \hbox{ if } i, i+1 \in A\hbox{ or }i, i+1 \notin A\\
\theta_{A \cup \{i+1\} \backslash \{i\}} & \hbox{ if } i\in A\hbox{ and }i+1 \notin A\\
\theta_{A \cup \{i\} \backslash \{i+1\}} & \hbox{ if } i+1\in A\hbox{ and }i \notin A
\end{cases}~.
\end{equation}
These operators instead of exchanging an $i$ for an $i+1$ like the symmetric group
action, have the effect of shifting the indices of the variables (if possible).  They
are therefore known as quasisymmetric operators.  They were studied in depth by
Hivert \cite{Hi}.  The operators are not multiplicative on $R_n$ in general since, for example,
\[
\pi_1( \theta_{1} \theta_{2})
= \theta_1 \theta_2
= - \pi_1( \theta_{1}) \pi_1(\theta_{2})~.
\]
They are also not multiplicative when they act on the polynomial ring
in commuting variables.

A polynomial that is invariant under the action of quasisymmetric operators
is said to be quasisymmetric invariant (or just `quasisymmetric').
The quasisymmetric invariants of $R_n$ are
linearly spanned by the elements:
\begin{equation}\label{eq:defF}
F_{1^r}(\theta_1, \theta_2, \ldots, \theta_n) := \sum_{\substack{A \subseteq [n]\\|A|=r}} \theta_A~.
\end{equation}
The symbols $F_{1^r}$ for the elements borrows the notation from the
polynomial ring in commuting variable invariants known as the `fundamental
quasisymmetric polynomials.'  The commuting polynomial quasisymmetric
invariants are indexed by compositions.

%%%%%%%%%%%%%%%%%%%%%%%%%%%%%%%%%%%%%%
\subsection{Quasisymmetric functions generate a commutative subalgebra}
In \cite{FLP}, the authors showed that the quasisymmetric functions in
one set of commuting variables and one set of anticommuting variables
forms a bi-graded Hopf algebra.  This implies that the quasisymmetric functions
in one set of anticommuting variables are closed under multiplication
and the space is spanned by one element at each non-negative degree.
The product is not commutative in general, but it is in the case of this subalgebra
even though the variables are anticommuting.

In the notation of \cite{FLP}, $F_{1^r}=M_{\dot{0}^r}=L_{\dot{0}^r}$ where $\dot{0}^r=(\dot{0},\dot{0},\ldots,\dot{0})$ a composition of length $r$. The fermionic degree of $F_{1^r}$ is exactly $r$.
In general, the elements of

\begin{prop} \label{prop:comm}
The subalgebra generated by quasisymmetric invariants $\{F_{1^r}|r\ge 0\}$ is commutative.
Moreover,
$F_{1^r} F_{1^s} = a_{r,s} F_{1^{r+s}}$, where for $r+s \leq n$,
\[a_{r,s} = \begin{cases}
	\left({\lfloor \frac{r+s}{2} \rfloor \atop \lfloor \frac{r}{2} \rfloor}\right)&\hbox{ if }rs \equiv 0~({\rm mod}~2).\\
	0&\hbox{otherwise}
\end{cases}~.
\]
\end{prop}
A remark brought to our attention by D. Grinberg \cite{} shows that
$a_{r,s}$ is equal to the $q$-binomial coefficient $\qbinom{\lfloor \frac{r+s}{2} \rfloor}{\lfloor \frac{r}{2} \rfloor}$ evaluated at $q \rightarrow -1$. \todo{is this right? Darij's remark hasn't made it into the paper yet and I need to find his email.}

\subsection{The ideal generated by symmetric invariants}
The symmetric invariants ${\rm Sym}_{R_n}$ of $R_n$ are very small since a basis consists of only two
elements $1$ and $F_1(\theta_1, \theta_2, \ldots, \theta_n)$.  Therefore the ideal generated by the invariants of non-zero
degree, which we shall denote $J_n$, is generated by a single element
$F_1$.
We begin by considering the symmetric
coinvariants of $R_n$, the quotient ring $R_n/J_n$.
Because the ideal $J_n$ is principal we can
understand this quotient with much more detail.
This quotient ring is a special case of the ring
recently studied in \cite{IRR,KR}.

Recall that ${\rm dim}~R_n = 2^n$,
and if we consider the quotient $R_n/J_n$ it is isomorphic to $R_{n-1}$ since
in this algebra $\theta_n = - \theta_1 -\theta_2 - \cdots - \theta_{n-1}$.
Let $A \subseteq [n-1]$ and $A' = A \cup \{n\}$, then
the map which sends
$\theta_{A'}$ to
$$-\theta_{A}(\theta_1 + \theta_2 + \cdots + \theta_{n-1}) \otimes 1
+ \theta_{A} \otimes F_1\quad \in\ R_n/J_n \otimes {\rm Sym}_{R_n}$$
and $\theta_{A}$ to
$$\theta_{A} \otimes 1 \quad \in \ R_n/J_n \otimes {\rm Sym}_{R_n}$$
is an algebra isomorphism.  Since this map describes the image
for each monomial in $R_n$, we have the following proposition.

\begin{prop} \label{prop:free}
For each $n \geq 1$,
$$R_n \cong R_n/J_n \otimes {\rm Sym}_{R_n},$$
as an algebra.  That is, $R_n$ is free over ${\rm Sym}_{R_n}$.
\end{prop}

%%%%%%%%%%%%%%%%%%%%%%%%%%%%%%%%%%%%%%
\subsection{The ideal generated by the quasisymmetric invariants}

Define an ideal of $R_n$ generated by the quasisymmetric invariants as
\[
I_n := \left< F_{1^r}(\theta_1, \theta_2, \ldots, \theta_n) : 1 \leq r \leq n \right>.
\]


The \emph{exterior quasisymmetric coinvariants}\footnote{We borrow
the name `coinvariant' space even though the generators, and not the whole ideal, is invariant under the quasisymmetric operators.} are defined to be $EQC_n := R_n/I_n$. Similar to its commutative variant, the ideal $I_{n}$ is not invariant under the action of $\pi$. So the quotient $EQC_{n}$ is not closed under the action of $\pi$.


%%%%%%%%%%%%%%%%%%%%%%%%%%%%%%%%%%%%%%
\subsection{Differential operators on the exerior algebra}\label{ssec:harm}
We can define a set of differential operators on $R_n$ which
will permit us to define the orthogonal complement to the
ideal and a notion of quasisymmetric harmonics.

The operators $\partial_{\theta_i}$ act on monomials in $R_n$
by
\[
\partial_{\theta_i}( \theta_A ) = \begin{cases}
(-1)^{\#\{ j \in A: j<i\}}\theta_{A \backslash \{i\}}&\hbox{ if }i \in A\\
0&\hbox{ if }i \notin A
\end{cases}~.
\]

The operators can equally be characterized by the action that $\partial_{\theta_i}(1) = 0$
and the commutation relations
\[
\partial_{\theta_i} \partial_{\theta_j}=-\partial_{\theta_j} \partial_{\theta_i}
\hbox{ if } 1 \leq i \neq j \leq n
\qquad\hbox{and}\qquad
\partial_{\theta_i}^2 = 0\hbox{ for }1 \leq i \leq n
\]
\[
\partial_{\theta_i} \theta_j=-\theta_j \partial_{\theta_i}
\hbox{ if } 1 \leq i \neq j \leq n
\qquad\hbox{and}\qquad
\partial_{\theta_i} \theta_i = 1\hbox{ for }1 \leq i \leq n~.
\]

For a monomial $\theta_A = \theta_{a_1} \theta_{a_2} \cdots \theta_{a_r}$,
let $\overline{\theta_A} = \theta_{a_r} \theta_{a_{r-1}} \cdots \theta_{a_1}$ represent
reversing the order of the variables in the monomial.  Extend this notation to both
differential operators and polynomials (and polynomials of differential operators)
by extending the notation linearly. 
We can define an inner product on $R_n$ by setting for $p,q \in R_n$.
\[
\left< p, q \right> = \overline{p(\partial_{\theta_1}, \partial_{\theta_2}, \ldots, \partial_{\theta_n})}
q( \theta_1, \theta_2, \ldots, \theta_n)|_{\theta_1=\theta_2 = \cdots=\theta_n=0}~.
\]
The monomials of $R_n$ form an orthonormal basis of the space with respect to this
inner product.

Define the orthogonal complement to $I_n$ with respect to the inner product as
the set
\begin{align}
EQH_n :&= \left\{ q \in R_n : \left< p, q \right> = 0 \hbox{ for all } p \in I_n \right\}
\label{eq:set1}\\
  &=\left\{ q \in R_n : p(\partial_{\theta_1}, \partial_{\theta_2}, \ldots, \partial_{\theta_n})
q= 0 \hbox{ for all } p \in I_n \right\}~.\label{eq:diffeqs}
\end{align}
The second equality follows from the fact that $I_n$ is an ideal
and shows that $EQH_n$ is also the solution space of
a system of differential equations.  
We refer to $EQH_n$ as the \emph{exterior quasisymmetric harmonics}.\footnote{
The harmonics and diagonal harmonics borrows the name from the physics literature
because the harmonic operator $\partial_1^2 + \partial_2^2 + \cdots + \partial_n^2$
is symmetric in the differential operators.  In the case of the exterior algebra,
this operator acts as zero and yet we persist by borrowing the name from the
analogous spaces of commuting variables.
} 
The inner product is positive definite. It follows that, as graded vector spaces, $
EQC_n \simeq EQH_n$ for all $n \ge 1$.

We will conclude this section  by constructing a set of linearly independent elements inside $EQH_n$, which will
give us a lower  bound on the dimension of $EQC_n$. In Section~\ref{sec:ballotbasis} we will see that this is also an upper bound,
thus concluding that our  set is in fact a basis. To compute $EQH_n$ we need to solve the differential equations in Equation~\eqref{eq:diffeqs}.
Remark first  that since $I_n$ is an ideal, we do not need to take all $p\in I_n$, but it is enough to solve for the generators $p=F_{1^r}$ for  $1\le r\le n$.
We can reduce that further using Proposition~\ref{prop:comm} as noted in the following lemma.
\begin{lemma}\label{lem:idealgen}
For $n\ge 2$ we have $I_n$ is the ideal generated by $F_1$ and $F_{1^2}$.
\end{lemma}

\begin{proof} Clearly we have that the ideal generated by $F_1, F_{1^2}$ is contained in $I_n$.
For the converse we note that for each $k\geq 1$
there are non-zero coefficients $a$ and $a'$ such that
\[
a F_{1^{2k}} = (F_{1^2})^k \hskip .2in\hbox{   and   }\hskip .2in a' F_{1^{2k+1}} = (F_{1^2})^k F_1
\]
hence all of the generators of $I_n$ are contained in the ideal generated by $F_1, F_{1^2}$.
%For the  converse, consider first $2\le r=2k\le n$. Using Proposition~\ref{prop:comm}, we have that
%$aF_{1^r} = \big(F_{1^2}\big)^k$ for some $a\ne 0$. Therefor $F_{1^r} \in \langle F_1, F_{1^2}  \rangle$. For $2\le r=2k+1\le n$, we have $aF_{1^r} = \big(F_{1^2}\big)^kF_1$, which again show $F_{1^r} \in \langle F_1, F_{1^2}  \rangle$. We conclude that $I_n\subseteq  \langle F_1, F_{1^2}  \rangle$.
\end{proof}
From this we conclude that
\begin{equation}\label{eq:defEQH}
EQH_n =  \Big\{ q \in R_n :  \quad\sum_{1\le i\le n} \partial_{\theta_i}q= 0 \quad\hbox{ and  }\quad \sum_{1\le i<j\le n} \partial_{\theta_j}\partial_{\theta_i}q= 0 \Big\}~.
\end{equation}

Given $0\le k\le \lfloor \frac{n}{2}\rfloor$, a non-crossing pairing of length $k$ is a
list $(C_1, C_2, \ldots, C_k)$ with
\begin{align*}
 &C_r=(i_r,j_r) \text{ for $1\le i_r<j_r\le n$ for each $1 \leq r \leq k$ and,}\\
 &\qquad \text{either } i_r<j_r<i_s<j_s  \text{ or }  i_s<i_r<j_r<j_s\,\text{ for any $1\le r<s\le k$}.
\end{align*}
Given a non-crossing pairing $C=(C_1,C_2,\ldots C_k)$, we define
\begin{equation}\label{eq:Deltadef}
\Delta_C = \big(\theta_{j_1}-\theta_{i_1}\big)\big(\theta_{j_2}-\theta_{i_2}\big)\cdots \big(\theta_{j_k}-\theta_{i_k}\big)\,.
\end{equation}
Here $\Delta_C=1$ if $k=0$. Remark that $j_1<j_2<\cdots< j_k$.
The following proposition shows that there is a relationship between the non-crossing
partition condition and the differential equations from Equation \eqref{eq:defEQH}.

\begin{prop}\label{prop:harmelem}
The set
$${\mathcal D}'_n =\big\{ \Delta_C: \text{ $C=(C_1,C_2,\ldots, C_k)$  non-crossing pairing and $0\le k\le \lfloor \frac{n}{2}\rfloor$}\big\}
$$
is contained in $EQH_n $.
\end{prop}


The set ${\mathcal D}'_n$ is not linearly independent, for example for $n=3$ and $k=1$, we have the following three non-crossing pairing:
$((1,2))$, $((1,3))$ and $((2,3))$, but
\[
\Delta_{((1,2))} - \Delta_{((1,3))} + \Delta_{((2,3))} =0 ~.
\]
We want to select a linearly independent subset of ${\mathcal D}'_n$. We proceed as follows:
consider a sequence
$\alpha = (a_1, a_2, \ldots, a_n) \in \{0, 1\}^n$
such that $\sum_{i=1}^r a_i \leq r/2$ for all $1 \leq r \leq n$.
Such sequences are known as \defn{ballot sequences}.
If ever it is the case that $\sum_{i=1}^r a_i > r/2$ then we say that
$\alpha$ \defn{breaks the ballot condition} at position $r$.

%In Section~\ref{sec:path} we will develop a more visual interpretation of
%ballot sequences and interpret them as paths that stay above the diagonal.
Given a ballot sequence $\alpha$ we build a non-crossing pairing $C(\alpha)$ by first replacing all $0$s
by open parentheses $0\mapsto$`(',
and all $1$s by close parentheses $1\mapsto$`)',
and then do the natural maximal pairing of parenthesis. The positions of the pairings
give us in lexicographic order a non-crossing pairing which we shall denote $C(\alpha)$.
Since $\alpha$ is a ballot sequence, every closed parenthesis is matched
and some open parentheses might remain unpaired.
The natural pairing of parenthesis guarantees that the result will be non-crossing. For example,
\[
\alpha=0010001101 \qquad\mapsto\qquad (()((())() \qquad\mapsto\qquad C(\alpha)=((2,3),(6,7),(5,8),(9,10)) \,.
\]

The total number of ballot sequences of size $n$ is equal to $\binom{n}{\lfloor{n/2}\rfloor}$
(see \cite[\href{https://oeis.org/A001405}{A001405}]{OEIS}).

Given this construction we have the following Proposition.

\begin{prop}\label{prop:harmbasis}
The set
$${\mathcal D}_n =\big\{ \Delta_{C(\alpha)}:  \alpha \in \{0, 1\}^n \text{ a ballot sequence}\big\}
$$
is contained in $EQH_n$ and is linearly independent.
\end{prop}

\begin{proof}
  The first statement follows from Proposition~\ref{prop:harmelem} since ${\mathcal D}_n \subseteq {\mathcal D}'_n \subseteq EQH_n$.
To show the linear independence, fix $\alpha$ a ballot sequence and let $C(\alpha)=((i_1,j_1),\ldots,(i_k,j_k))$ be its non-crossing pairing.  We remark that the sequence of numbers
$j_1<j_2<\cdots<j_k$ corresponds to the position of the $1$s in $\alpha$.
Using the monomial ordering described in Section \ref{sec:linbasis}
and by inspection of the product in Equation~\eqref{eq:Deltadef},
we observe that the term $\theta_{j_1}\theta_{j_2}\cdots\theta_{j_k}$ is the smallest lexicographic
monomial in $\Delta_{C(\alpha)}$.
For different ballot sequences $\alpha$ we get different positions of the $1$s in
$\alpha$ and thus different smallest lexicographic monomials,
which shows the independence of ${\mathcal D}_n$.
\end{proof}

Ballot sequences with $k$ number of $1$'s are in bijection with standard tableaux of shape $(n-k,k)$. It is tempting to try describe $\mathcal{D}_{n}$ in terms of polynomials directly analogous to Specht polynomials in commuting variables. However, it can be shown that such a direct analogy does not work.

%%%%%%%%%%%%%%%%%%%%%%%%%%%%%%%%%%%%%%
%%%%%%%%%%%%%%%%%%%%%%%%%%%%%%%%%%%%%%
%%%%%%%%%%%%%%%%%%%%%%%%%%%%%%%%%%%%%%
\section{A linear basis of the ring}\label{sec:linbasis}

Again let $n$ be a fixed positive integer and $R_n = {\mathbb Q}[\theta_1, \theta_2, \ldots, \theta_n]$.
We have thus far represented the basis for
$R_n$ as the elements $\theta_A$ with $A \subseteq [n]$.  Define $\alpha(A) \in \{ 0,1\}^n$ to be
the sequence $a_1 a_2 a_3 \cdots a_n$ with $a_i = 1$ if $i \in A$ and
$a_i = 0$ if $i \notin A$ so that
\[
\theta_A = \theta_1^{a_1} \theta_2^{a_2} \cdots \theta_n^{a_n} := \theta^{\alpha(A)}~.
\]
For such a sequence $\alpha \in \{0,1\}^n$, let $m_1(\alpha) := \sum_{i=1}^n a_i$
represent the number of $1$s in the string.  This will also be the degree of the monomial
$\theta^{\alpha}$.

For sequences $\alpha \in \{ 0, 1 \}^n$, define elements $G_\alpha$ by
\begin{equation}\label{eq:Gdef1}
G_{1^s0^{n-s}} = F_{1^s}
\end{equation}
and if $\alpha \neq 1^s 0^{n-s}$, then $\alpha$ is of the form $u01^s0^{n-k-s}$ for some string $u$
of length $k-1$ and we recursively define
\begin{equation}\label{eq:Gdef2}
G_{u01^s0^{n-k-s}} = G_{u1^s0^{n-k-s+1}} - (-1)^{m_1(u)} \theta_k G_{u1^{s-1}0^{n-k-s+2}}~.
\end{equation}

We will show below that the recurrence for the
$G_\alpha$ is defined so that they are $S$-polynomials \cite{CLO}
for elements of the ideal $I_n$.
In commutative variables, similar polynomials were defined by Aval-Bergeron-Bergeron~\cite{AB,ABB}
as a (complete) subset of $S$-polynomials needed to compute all possible
$S$-polynomials in the Buchburger algorithm for a Gr\"obner basis.
It is not given that one can easily describe such a set of $S$-polynomials and here we have adapted
the definition for working in the exterior algebra.

\begin{example} For $\alpha = 010110$ and $\beta = 001100$, we compute
the elements $G_\alpha$ and $G_\beta$ using the definition.
\begin{align*}
G_{010110} &= G_{011100} + \theta_3 G_{011000} = (G_{111000} - \theta_1 G_{110000})
+ \theta_3(G_{110000} - \theta_1 G_{100000})\\
&= \theta_2 \theta_4 \theta_5 + \theta_2 \theta_4 \theta_6 + \theta_2 \theta_5 \theta_6
+ 2 \theta_3 \theta_4 \theta_5 + 2 \theta_3 \theta_4 \theta_6 + 2 \theta_3 \theta_5 \theta_6
+ \theta_4 \theta_5 \theta_6
\end{align*}

and we have that
\begin{align*}
G_{001100} &= G_{011000} - \theta_2 G_{010000} = (G_{110000} - \theta_1 G_{100000})
- \theta_2 (G_{100000} - \theta_1 G_{000000})\\
&= \theta_3 \theta_4 + \theta_3 \theta_5 + \theta_3 \theta_6 + \theta_4 \theta_5 + \theta_4 \theta_6 + \theta_5 \theta_6.
\end{align*}
\end{example}

\begin{prop}\label{prop:largest}
The largest lexicographic term in $G_\alpha$ is $\theta^\alpha$.
\end{prop}

The proof of Proposition \ref{prop:largest} follows by induction
on the length of $\alpha$ and
from a lemma that is analogous to Lemma 3.3 of \cite{AB}.
The recursion in this result is really the origin of the definition
of $G_\alpha$ because Equation \eqref{eq:Gdef2}
was adapted so that this lemma holds.  It follows that
the set $\{ G_\alpha \}_{\alpha \in \{0,1\}^n}$ is a basis
for $R_n$.


%%%%%%%%%%%%%%%%%%%%%%%%%%%%%%%%%%%%%%
%%%%%%%%%%%%%%%%%%%%%%%%%%%%%%%%%%%%%%
%%%%%%%%%%%%%%%%%%%%%%%%%%%%%%%%%%%%%%
\section{A basis for the quotient}\label{sec:ballotbasis}

%To each sequence $\alpha \in \{0, 1\}^n$, we associate a path starting at the origin
%and extending into the first quadrant of the $x,y$-plane.  The $i^{th}$ step of this
%path will be a unit in the $(1,0)$-direction if $a_i =1$ and it will be a unit in the $(0,1)$-direction
%if $a_i = 0$.  We say that the sequence $\alpha$ \defn{crosses the diagonal} if
%there is a point on the path which lies at $(a,a)$ and the next step is in the $(1,0)$
%direction.  Otherwise we say that $\alpha$ \defn{stays above the diagonal}.  Note that
%$\alpha$ stays above the diagonal if $\sum_{i=1}^r a_i \leq r/2$
%for all $1 \leq r \leq n$
%and it crosses the diagonal otherwise. The ballot sequences we encountered earlier are the one that  stay above the diagonal.
%\mike{I think that we should state everything in terms of ballot sequences
%or paths, but not both.  I see no obvious reason to switch between the two models
%and we should make the presentation as simple as possible.}
%
%\begin{example} Let $n=6$ and $\alpha = 010110$ and $\beta = 001100$, then the corresponding
%paths are
%\begin{center}
%\begin{tikzpicture}[scale=.75]
%  \draw[dotted] (0, 0) grid (4, 4);
%  \draw[rounded corners=1, color=black, line width=2] (0, 0) -- (0, 1) -- (1, 1) -- (1, 2) -- (2, 2) -- (3, 2) -- (3, 3);
%\end{tikzpicture}
%\hskip .5in
%\begin{tikzpicture}[scale=.75]
%  \draw[dotted] (0, 0) grid (4, 4);
%  \draw[rounded corners=1, color=black, line width=2] (0, 0) -- (0, 1) -- (0, 2) -- (1, 2) -- (2, 2) -- (2, 3) -- (2, 4);
%\end{tikzpicture}
%\end{center}
%\end{example}

The elements $G_\alpha$ are defined so that we could use them to
identify a nice basis of the ideal $I_n$. This is our main theorem.

\begin{theorem}\label{thm:basisofideal}
The set $A_n:=\big\{ G_\alpha : \alpha \in \{0,1\}^n \text{ \it breaks the ballot condition}\big\}$
is a $\mathbb Q$--linear basis of the ideal $I_n$.
\end{theorem}

The proof of this theorem uses our understanding of the harmonic space $EQH_n\cong EQC_n$.
In Proposition~\ref{prop:harmbasis} we found that $\dim(EQH_n)=\dim(EQC_n)$ is at least the number of ballot sequences.
We first establish a small lemma about a spanning set for the quotient $EQC_n$ showing that  the dimension is at most  the number of ballot sequences.
Therefore we have equality
and the set ${\mathcal D}_n$ in Proposition~\ref{prop:harmbasis} is in fact a basis
of $EQH_n$.

There are several straightforward consequence of this theorem which we state here.

\begin{cor}\label{cor:hilb}  The Hibert series of $EQH_{n}$ is given by Equation~\eqref{eq:hilb}.
\end{cor}

\begin{cor} The set ${\mathcal D}_n$ is a basis of $EQH_n$ and of $EQC_n$.
\end{cor}

\begin{cor}\label{cor:minimalGB}
The set $A_n$ is a (non-reduced, non-minimal) Gr\"obner basis of $I_n$. A minimal Gr\"obner basis for $I_n$ is given by
$$ I_n =\big\{ G_\alpha : \alpha \in \{0,1\}^n \text{ \it breaks the ballot condition only
at the rightmost  1 of }  \alpha\big\}\,.
$$
\end{cor}

\printbibliography 
\end{document}
